%
% GNU courseware, XIN YUAN, 2018
%

\section{C++程序的优化}

\frame{
\centerline{\textbf{\Huge{C++程序的优化}}}
}

\frame{\frametitle{原则}
\begin{itemize}
\item<1-> \textbf{占用内存空间},包括程序集文件大小、函数调用层次、
局部变量栈的占用、堆的使用。
\item<2-> \textbf{运行时间},即程序的运算速度。
\end{itemize}
}

\frame{\frametitle{面向空间}
以空间换时间,就是把大量变化较小的数据预先算好存放在连续的内存中,随时取用,
总体上减少了运算的指令数目。
}

\frame{\frametitle{面向空间}
\begin{itemize}
\item<1-> 对大量小对象,分配大块连续内存,并使用自定义分配策略。
\item<2-> 使用free-list技术来写数据结构容器类。
\item<3-> 内存池、对象池、线程池、任务池。
\end{itemize}
}

\frame{\frametitle{面向空间}
\begin{itemize}
\item<1-> 预先计算的查找表技术。
\item<2-> 生产者-消费者应用场景下,使用双缓冲技术,直接交换缓冲指针。
\end{itemize}
}

\frame{\frametitle{面向时间}
从算法本身入手,在总体上减少运算的指令数目。
}

\frame{\frametitle{面向时间}
\begin{itemize}
\item<1-> 变换问题空间。
\item<2-> 把问题描述成具有启发性的几何形状。
\item<3-> 递推和迭代方法。
\end{itemize}
}

\frame{\frametitle{面向时间}
\begin{itemize}
\item<1-> 问题降解。
\item<2-> 充分利用算法中存在的对称性。
\item<3-> 分治和动态规划方法。
\end{itemize}
}

\frame{\frametitle{面向时间}
\begin{itemize}
\item<1-> 挖掘硬件性能,使用多核CPU和GPU进行并行计算。
\item<2-> 使用位运算加速某些整数的运算。
\item<3-> 尽可能避免使用if语句和for语句,使用循环展开等技术。
\end{itemize}
}

\frame{\frametitle{面向时间}
\textbf{例:矩阵运算加速}

~

采分治和分解的思想,将矩阵分块,并调整循环变量先后执行的顺序,
使得分块所在的连续内存能够缓存命中。
同时将计算拆解成多次迭代,对矩阵同一位置的计算不是一次完成的,
而是多轮迭代后完成,从而在整体上减少需要运算的指令总数。
}

\frame{\frametitle{参考书}
{\CJKfamily{zhkai}
\begin{itemize}
\item<1-> 《计算机程序设计艺术》(经典古老)
\item<2-> 《高效程序的奥秘》
\item<3-> 《代码编程之美》
\end{itemize}
}
}

%end
